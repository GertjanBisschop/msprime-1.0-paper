\documentclass{article}
\usepackage[round]{natbib}
\usepackage{listings}
\usepackage[english]{babel}%
\usepackage[T1]{fontenc}%
\usepackage[utf8]{inputenc}%
\usepackage{amsmath,amssymb,amsfonts}%
\usepackage{geometry}%
\usepackage{dsfont}
\usepackage{verbatim}%
\usepackage{environ}%
\usepackage[right]{lineno}%
\usepackage{showkeys}

% local definitions
\newcommand{\msprime}[0]{\texttt{msprime}}
\newcommand{\tskit}[0]{\texttt{tskit}}
\newcommand{\ms}[0]{\texttt{ms}}
\newcommand{\scrm}[0]{\texttt{scrm}}
\newcommand{\stdpopsim}[0]{\texttt{stdpopsim}}

%% local definitions for section multiple merger coalescents
\newcommand{\msprime}[0]{{\texttt{msprime} }}
\newcommand{\tskit}[0]{{\texttt{tskit} }}
\newcommand{\ms}[0]{{\texttt{ms} }}
\newcommand{\scrm}[0]{{\texttt{scrm} }}
\newcommand{\stdpopsim}[0]{{\texttt{stdpopsim} }}
 \newcommand{\be}{\begin{equation}}
 \newcommand{\ee}{\end{equation}}
 \newcommand{\bd}{\begin{displaymath}}
 \newcommand{\ed}{\end{displaymath}}
\newcommand{\IN}{\ensuremath{\mathds{N}}}%
\newcommand{\EE}[1]{\ensuremath{\mathds{E}\left[ #1 \right]}}%
\newcommand{\one}[1]{\ensuremath{\mathds{1}_{\left\{ #1 \right\}}}}%
\newcommand{\prb}[1]{\ensuremath{\mathds{P}\left( #1 \right) } }%

\NewEnviron{esplit}[1]{%
\begin{equation}
\label{#1}
\begin{split}
  \BODY
\end{split}\end{equation}
}



\begin{document}

\title{msprime 1.0: an efficient and extensible coalescent simulation framework}
\author{Author list to be filled in
}
% \address{

% \section{Contact:} \href{jerome.kelleher@bdi.ox.ac.uk}{jerome.kelleher@bdi.ox.ac.uk}

\maketitle

% Abstract: ~ 150 words.  Concise summary of the results.

% Key points in abstract.
% 1) Coalescent simulation is a vital tool
% 2) Coalescent with recombination is hard
% 3) Large sample sizes are crucial today
% 4) Msprime has lots of features and uses a community development model

\begin{abstract}
Coalescent simulation is a key tool in population genetics and
and has been integral to coalescent theory since its earliest days.
Because of the ease at which the basic model can be simulated,
a large number of different simulators have been developed. However,
the coalescent with recombination is far more challenging to simulate,
and, until recently, it was not possible to simulate efficiently.
The \msprime\ software has revolutionised
coalescent simulation, making it possible to simulate millions
of whole genomes for the first time. We summarise the many features
of \msprime\ 1.0, which is built around a core model of efficiently
implementing recombination via the succinct tree sequence data
structure. We advocate a community oriented open-source development
process as a way to reduce duplication of effort and increase
the quality of simulation software.
\end{abstract}

% Keywords: For software, include the relevant models, algorithms and language;
\textbf{Keywords:} Coalescent simulation, Python

%%%%%%%%%%%%%%%%%%%%%%%%%%%%%%%%%%%%%%%%%%%%%%%%%%%%%%%%%%%%%%%%%%%%%%%%%%%%%%%%%%%%%
% Introduction: Limit of 500 words. The introduction should describe the
% significance of the software or resource presented, discuss novelty, provide an
% overview of the software or resource, and describe how researchers can access it.
%%%%%%%%%%%%%%%%%%%%%%%%%%%%%%%%%%%%%%%%%%%%%%%%%%%%%%%%%%%%%%%%%%%%%%%%%%%%%%%%%%%%%

\section*{Introduction}

The coalescent
process~\citep{kingman1982coalescent,hudson1983testing,tajima1983evolutionary}
models the ancestry of set of sampled genomes. Given some sampled
genomes, the coalescent provides a mathematical description of the
genealogical tree relating these to each other. Because of its
elegance and simplicity, the coalescent is now central to much of
population genetics and
genomics~\citep{hudson1990gene,hein2004gene,wakely2008coalescent}.
Simulation has been a vital
tool in coalescent theory since its earliest days~\citep{hudson1983testing},
and dozens of different simulation tools have been
developed over the decades~\citep{carvajal2008simulation,liu2008survey,
arenas2012simulation,yuan2012overview,hoban2012computer}.
Two developments of recent years, however, have made the majority
of these simulators of little relevance to modern datasets.
% Some citations here? Darwin tree of life maybe?
Firstly, whole genome sequencing technology---and in particular fourth-generation
sequencing technologies---have made chromosome-level assemblies
commonplace. Thus, regarding input data as a series of unlinked loci
(a previously reasonable assumption), is no longer defensible and
recombination must be fully accounted for. While the
single-locus coalescent can be simulated in constant
time~\citep{hudson1990gene}, the coalescent with recombination is
far more challenging to simulate.
% Sentence is too long...
Programs such as Hudson's classical \ms~\citep{hudson2002generating}
can only simulate short segments under the influence of recombination,
and, until recently, we had to resort to
approximate models of recombination~\citep{mcvean2005approximating,staab2014scrm}
if we wished to simulate whole chromosomes.

The second development that has made the majority of
present day simulators inapplicable to modern data sets is the
exponential increase in sample size. Human datasets such
UK Biobank~\citep{bycroft2018genome} and
gnomAD~\citep{karczewski2019variation} now contain hundreds of
thousands of genomes, and there is every reason to believe that
datasets of millions will soon be commonplace~\citep{stephens2015big}.
Classical simulators such as \ms\ and even fast approximate simulators
such as \scrm\ simply cannot cope with this scale. Even if we
spend months of CPU time on running simulations with millions of samples,
the output would require terabytes of storage space to store and
days of CPU time to parse~\citep{kelleher2016efficient}.

The \msprime\ simulator~\citep{kelleher2016efficient,kelleher2020coalescent}
has profoundly changed
what is possible. Through the use of efficient data structures, it is
now possible to simulate millions of whole chromosomes, without resorting
to approximations. The ``succinct tree
sequence'' data
structure~\citep{kelleher2016efficient,kelleher2018efficient,kelleher2019inferring},
introduced as part of \msprime\, makes it possible to store simulations
of millions of whole chromosomes in a few gigabytes, several orders
of magnitude smaller than the standard
Newick~\citep{felsenstein1989phylip} formats. This data structure has
also lead to major advances in forwards-time
simulation~\citep{kelleher2018efficient,haller2018tree},
ancestry inference~\citep{kelleher2019inferring}
and calculation of population genetic statistics~\citep{ralph2019efficiently}.
Through a rigorous open-source community development process,
\msprime\ has gained a large number of features since its introduction,
making a highly efficient and flexible platform for population
genetic simulation.
% This is a bit flat --- anything else to say at the end here?
This paper marks the release of \msprime\ 1.0.

% JK: Earlier notes on this para
% - Msprime 1.0 combines the features of many different simulators and is
%   far more efficient than all of them. The combination of a Python API,
%   an extremely efficient storage format, and a well-engineered and
%   extensible codebase changes the game. Stdpopsim builds on this base
%   to provide easy access to standard models of population genetics.
%


%%%%%%%%%%%%%%%%%%%%%%%%%%%%%%%%%%%%%%%%%%%%%%%%%%%%%%%%%%%%%%%%%%%%%%%%%%%%%%%%%%%%%
% Methods: For software, the authors should describe key elements of the algorithms
% implemented, and are expected to provide both the source code as well as
% examples illustrating the use of the software.
%%%%%%%%%%%%%%%%%%%%%%%%%%%%%%%%%%%%%%%%%%%%%%%%%%%%%%%%%%%%%%%%%%%%%%%%%%%%%%%%%%%%%

\section*{Methods}

In this section we describe the main features of \msprime\ 1.0.

\subsection*{Recombination and gene conversion}
Recombination is implemented in \msprime\ via
Hudson's algorithm, which
works backwards in time tracking the
through the effects of common ancestor and recombination events
~\citep{hudson1983properties,hudson1990gene,kelleher2016efficient}. Common
ancestor events combine the ancestral material of two lineages, and may
 result in coalescences in the marginal trees. Recombination events
split the ancestral material for some lineage, creating two independent
lineages. Using the appropriate data structures~\citep{kelleher2016efficient},
this process is much more efficient to simulate than either
methods based on the Ancestral Recombination Graph
formulation~\citep{griffiths1991two,griffiths1997ancestral}
or the left-to-right approach~\citep{wiuf1999recombination,wiuf1999ancestry}.


We also support variable recombination rates along a chromosome.
This allows us to model both recombination hotspots like
\texttt{msHOT}~\citep{hellenthal2007mshot} and also empirical
recombination maps like those supported by X Y Z.
(What about ~\cite{wang2014new}?)
We have also implemented gene
conversion following the model defined by~\cite{wiuf2000coalescent}
and used in \ms.
% % Uncomment this sentence, when the corresponding issue in the msprime github repo is resolved.
% Furthermore, msprime is now able to simulate gene conversion events that are
% initiated proportional to the local recombination rate specified by the recombination map.

Gene conversion is particularly useful to model homologous recombination in bacterial evolution.
We compare the performance of \msprime\ with gene conversion to \ms\ and
the whole bacterial genome simulator SimBac~\citep{brown2016simbac}.
Further features are planned to model bacterial evolution. In particular, we
aim to include bacterial gene transfer and utilize the succinct tree sequence structure
to represent the resulting ancestral gene transfer graph~\citep{baumdicker2014AGTG}.

\subsection*{SMC approximations}

The SMC~\citep{mcvean2005approximating} and SMC'~\citep{marjoram2006fast}
models are supported.
The performance of msprime has made the SMC
approximation largely unnecessary, at least when simulating human-like
recombination rates. The SMC is an important analytical approximation
forming the basis for several important inference methods
methods~\citep{
li2011inference,
harris2013inferring,
sheehan2013estimating,
schiffels2014inferring,
carmi2014renewal,
rasmussen2014genome,
zheng2014bayesian,
terhorst2017robust},
so simulations are still useful to examine  how good the approximation
is.

The SMC has been shown to be a good approximation to the
coalescent with recombination for samples of less than
four~\citep{hobolth2014markovian,wilton2015smc}
and in simulations, produces patterns of LD indistinguishable from the
full coalescent~\citep{mcvean2005approximating,marjoram2006fast}. However,
LD is a crude approximation of the correlation structure of trees
and it is not clear how good the SMC is for larger sample sizes.


\subsection*{Population structure}
We support all the stuff that ms does.
Demographic events where we change
the properties of the population.

We provide a debugger interface which allows the user to see the
various population sizes and growth rates over time. This includes
a way to compute coalescence rate trajectory, which provides a useful
ground truth when comparing with methods that estimate population
size over time~\citep{adrion2019community}.

The interface
is very flexible and general. We also reading in population tree
descriptions as produced by programs such as
StarBeast~\citep{heled2009bayesian}.

We support tracking lineages through time by associating a population
with each node, and having migration records. We also have
census events, which will record the location of every lineage
at a particular time.

\subsection*{Selection}
Selection was added to the coalescent ... Lots of simulators exist:
SelSim~\citep{spencer2004selsim}
mbs~\citep{teshima2009mbs},
msms~\citep{ewing2010msms},
% What about Cosi, did it simulate selection?
Cosi2~\citep{shlyakhter2014cosi2},
and discoal~\citep{kern2016discoal}.

We have implemented the basic infrastructure exists for the structured
coalescent, which makes adding support for, e.g.,
inversions straightforward~\citep{peischl2013sequential}


\subsection*{Discrete time Wright-Fisher}
The coalescent is a good approximation for ancient history, but
is bad for the recent past and in particular when we have large
sample
sizes~\citep{wakeley2012gene,bhaskar2014distortion,nelson2020accounting}.
Describe basic
approach [DOM, can you fill in please].
See ~\cite{nelson2020accounting} for more details.
We compare the performance of this simulation in Fig X
with ARGON~\citep{palamara2016argon}, a specialised DTWF simulator.

We support changing the simulation model, so that we can easily
run hybrid simulations like described by~\cite{bhaskar2014distortion}
This ia a powerful approach because it lets us combine any
number of different simulation models.

\subsection*{Integration with forwards-time simulations}
Forward simulations can also use tree
sequences~\citep{kelleher2018efficient,haller2018tree}. In particular,
SLiM 3.0~\citep{haller2019slim} and fwdpp~\citep{thornton2014cpp} can now output tree sequences.
Msprime can take tree sequences simulated by these forward simulators
as input and complete the simulation (``recapitation'').

\subsection*{Multiple merger coalescents}
%%Kingman's coalescent assumes that all mergers are binary. This
%%assumption can be violated in some situations. Lambda coalescents
%%~\citep{donnelly1999particle,schweinsberg2000coalescents}
%%HybridLambda~\citep{zhu2015hybrid}
%%We support X Y Z.

\label{mmc}%


Kingman's coalescent assumes that at most two ancestral lineages can
merge at each merger event.  This follows from the fact that the
Kingman coalescent can be shown to be the process describing the
random ancestral relations of gene copies sampled from a population
evolving according to the Wright-Fisher model (or similar models) of
genetic evolution.  Multiple-merger coalescent models can also be
obtained from appropriate extensions of the Moran
model \citep{EW06,HM12}, i.e.\ where a single randomly picked individual
(e.g.\ in a haploid population), or a pair of diploid parent
individuals \citep{BBE13}, produces offspring in each reproduction
event.  Convergence to the Kingman coalescent follows from certain
assumptions on the offspring distribution \citep{S99,MS01,SW08}.  These
assumptions may be violated in certain highly fecund
organisms \citep{hedgecock_94,B94,HP11,A04,irwin16}.  We take `high
fecundity' to mean the ability of individuals to produce numbers of
offspring on the order of the population size.  A key issue regarding
high fecundity involves the concept of `sweepstakes
reproduction' \citep{hedgecock_94}, where a few individuals may produce
the bulk of the offspring in any given generation.  Sweepstakes
reproduction is not captured by the Wrigth-Fisher model (or similar
models).  Population models, which do capture high fecundity and
sweepstakes reproduction, are in the domain of attraction of
`multiple-merger' coalescents \citep{DK99,P99,S99,S00}, in which a
random number of ancestral lineages may merge at any given time.  The
term `multiple merger' refers to a merger of at least three ancestral
lineages, or the simultaneous \citep{S00,MS01} merger of at least four
lineages in at least two distinct groups, and each group involving at
least two lineages.  Multiple-merger coalescent processes have also
been shown to be relevant for modeling the effects of selection on
gene genealogies \citep{Gillespie909,DS04}. 


Multiple-merger coalescents, in which a random number of ancestral
lineages may merge at a given time in one group, and only one such
group of lineages can merge at any given time (asynchronous multiple
mergers), are referred to as $\Lambda$-coalescents; the rate at which  a given group of $k$ lineages out of a total of  $b$ lineages merges  is given by ($a\ge  0$ constant)
\begin{equation}\label{lambdabk}
\lambda_{b, k} =  \int_0^1  x^{k-2}(1-x)^{b-k}\Lambda(dx) + a\one{k=2}, \quad 2 \le k \le b, 
\end{equation}
where $\Lambda$ is a finite measure on the  (Borel subsets of)  unit interval without an atom at zero \citep{DK99,P99,S99}.    The  total rate  is given by
\be\label{lambdab}
 \lambda_{b} = \int_0^1 \left(1 - (1-x)^{b} - bx(1-x)^{b-1} \right)x^{-2}\Lambda(dx) + \binom{b}{2},
\ee
\citep{S99}  which can be useful in applications, in particular  provided the  antiderivative  can be explicitly identified.     

A larger class of multiple-merger coalescents involving simultaneous
multiple mergers of distinct groups of ancestral lineages also exists
\citep{S00}. These are commonly referred to as $\Xi$-coalescents, and
can be shown to be the limits of ancestral processes derived from
population models incorporating diploidy (or more general polyploidy)
\citep{BBE13,Blath2016}, and certain models of selection \citep{DS04}.
To describe a general $\Xi$-coalescent let $\Delta$ denote the
infinite simplex \be\label{Delta} \Delta := \{ (x_1, \ldots ): x_1 \ge
x_2 \ge \cdots \ge 0, \sum_{j}x_j \le 1\}; \ee for any
$r \in \{1,2, \ldots\}$ let $k_1, \cdots, k_r \ge 2$, and
$b = s + k_1 + \cdots + k_r$ be the total number of blocks in a given
partition ($s = b - k_1 - \cdots - k_r$ is the number of blocks not
participating in the mergers at the given time).  The existence of
simultaneous multiple-merger coalescents was proved by \cite{S00}.
There exists a finite measure $\Xi$ on $\Delta$, with
$\Xi = \Xi_0 + a\delta_0$, the measure $\Xi_0$ has no atom at zero,
$a >0$ is fixed, and the rate at which one sees a given (simultaneous)
merger of ancestral lineages with merger sizes $k_1, \ldots, k_r$,
$s = n - k_1 - \cdots - k_r$, is given by
\begin{esplit}{xi}
  \lambda_{n; k_1, \ldots, k_r; s}  & = \int_\Delta  \sum_{\ell = 0}^s \sum_{\substack {i_1, \ldots, i_{r+\ell} = 1\\ \text{all distinct}} }^\infty  \binom{s}{\ell} x_{i_1}^{k_1} \cdots  x
_{i_{r}}^{k_r} x_{i_{r+1}} \cdots x_{i_{r+\ell}}\left(1 - \sum_{j=1}^\infty x_j \right)^{s-\ell} \frac{1}{ \sum_{j=1}^\infty x_j^2 } \Xi_0(dx)   \\
  & +  a\one{r=1, k_1 = 2}.
\end{esplit}%
The number of such $(k_1, \ldots, k_r)$ mergers is
\be\label{N}
    \mathcal{N}(b; k_1, \ldots, k_r ) = \binom{b}{k_1 \ldots k_r\, s} \frac{1}{ \prod_{j=2}^b\ell_j!  },
\ee
\citep{S00},   in particular  $\mathcal{N}(b;2) = b(b-1)/2$, and one can compute the total rate of a  $(k_1, \ldots, k_r)$ merger as
\be\label{lambdabkall}
      \lambda(n; k_1, \ldots, k_r)         =    \mathcal{N}(b; k_1, \ldots, k_r ) \lambda_{n; k_1, \ldots, k_r; s}.
\ee

The total rate is given by, with 
$n\ge 2$ denoting the total  number of ancestral lineages,  
\be
  \label{Xilambdab}
  \lambda_{n} = \int_\Delta \left(1 - \sum_{\ell = 0}^n \sum_{i_1 \neq
  \cdots \neq i_\ell } \binom{n}{\ell} x_{i_1}\cdots x_{i_\ell}\left(1
  - \sum_{i=1}^\infty x_i\right)^{n-\ell} \right)
  \frac{1}{\sum_{j=1}^\infty x_j^2}\Xi_0(dx) + a\binom{n}{2} \ee
  \citep{S00}.  Viewing coalescent processes strictly as mathematical
  objects, it is clear that the class of $\Xi$-coalescents contains
  $\Lambda$-coalescents as a specific example (i.e.\ allowing at most
  one group of lineages to merge each time), and the class of
  $\Lambda$-coalescents contain the Kingman-coalescent as a special
  case.  However, viewed as limits of ancestral processes derived from
  specific population models they are not nested, since one would
  obtain $\Lambda$-coalescents when deriving coalescent processes from
  haploid population models incorporating sweepstakes reproduction and
  high fecundity, and $\Xi$-coalescents for diploid populations.  One
  should therefore apply the models as appropriate, i.e.\
  $\Lambda$-coalescents to data (e.g.\ mtDNA data) inherited in a
  haploid fashion, and $\Xi$-coalescents to e.g.\ autosomal data
  inherited in diploid or polyploid fashion \citep{Blath2016}.  



In \msprime we have incorporated two examples of multiple-merger
coalescents.  One is a diploid extension \citep{BBE13} of the haploid
model of sweepstakes reproduction considered by \cite{EW06}, which is
a haploid Moran model adapted to sweepstakes reproduction.  Let $N$
denote population size, and take $\psi \in (0,1]$ to be fixed.  In
every generation, with probability $1-\varepsilon_N$ a single
individual (picked uniformly at random) perishes, and one of the
surviving individuals (sampled uniformly at random) produces one
offspring; with probability $\varepsilon_N$ a total of
$\lfloor \psi N \rfloor$ individuals perish, and of the remaining
individuals a single individual produces $\lfloor \psi N \rfloor -1 $
offspring.  Taking $\varepsilon_N = 1/N^\gamma$ for some $\gamma > 0$,
\cite{EW06} obtain specific examples of $\Lambda$-coalescents, where
the $\Lambda$ measure in Eq \eqref{lambdabk} is a point mass at
$\psi$.  The simplicity of this model does allow one to obtain some
explicit mathematical results (see e.g.\
\cite{EF2018,Matuszewski2017,Der2012,Freund2020}). The model
considered by \cite{EW06} has also been applied in algorithms for
simulating gene genealogies within phylogenies \citep{Zhu2015}. The
 specific model incorporated into \msprime is the diploid version
\citep{BBE13} of the model studied by \cite{EW06}, which would be
necessary in order to incorporate recombination.  In the model
considered by \cite{BBE13}, a single pair of diploid individuals
contribute offspring in each generation, selfing is excluded, and each
offspring is assigned one chromosome from each parent. There are,
therefore, four parent chromosomes involved in each reproduction
event, which can lead to up to four simultaneous mergers.  Let
\begin{esplit}{Cconst}
C_{b; k_1, \ldots, k_r;s } &=  \frac{4}{\psi^2} \sum_{\ell = 0}^{s \wedge (4-r)} \binom{s}{\ell} (4)_{r+\ell} (1-\psi)^{s-\ell }\left( \frac{\psi}{4} \right) ^{k_1 + \cdots + k_r + \ell},  \\
\end{esplit}
where  $C_{b; k_1, \ldots, k_r;s }$ corresponds to the coalescence rate $ \lambda_{b;k_1, \ldots, k_r;s}$ (see Eq. \eqref{xi}) of a  $\Xi$-coalescent on  $(\psi/4, \psi/4, \psi/4, \psi/4, 0, \ldots)$.
The total merger rate (see Eq.\ \eqref{lambdabkall}) is then 
\be\label{xidirlambdabk}
      \lambda(b;k_1, \ldots, k_r) =    \mathcal{N}(b; k_1, \ldots, k_r ) \left( \frac{c\psi^2/4}{1 +  c\psi^2/4}C_{b; k_1, \ldots, k_r;s } +     \frac{ \one{r=1, k_1 = 2} }{1 +  c\psi^2/4}  \right),
\ee
and the total coalescence rate (see Eq.\ \eqref{xilambdab}) becomes
    \begin{esplit}{xilambdab}
\lambda_b & =  \frac{c\psi^2/4}{1 +  c\psi^2/4}  \frac{4 }{\psi^2 } \left(1 - \sum_{\ell = 0 }^{b\wedge 4} \binom{b}{\ell} (4)_\ell \left( \frac{\psi}{4} \right)^\ell (1 - \psi)^{b-\ell } \right) +    \frac{1}{1 +  c\psi^2/4} \binom{b}{2}.  \\
\end{esplit}
The interpretation of Eq.\ \eqref{xilambdab} is that with probability  $1/(1 + c\psi^2/4)$  a `small' reproduction event occurs, in which the parent pair produce one  diploid offspring; with probability  $c\psi^2/4/(1 + c\psi^2/4)$ a `large' reproduction event occurs, in which  the parent pair produce  $\lfloor \psi N \rfloor$ offspring.   


The other multiple-merger coalescent model incorporated in \msprime is
derived from an adaptation of the haploid population model considered
by \cite{schweinsberg03} to diploid populations \citep{BLS15}.  In the
haploid version, in each generation individuals independently produce
random numbers of juveniles, where the random  number of juveniles $(X)$ produced by
a given individual  has the stable law \be\label{jX} \lim_{k\to
\infty} C k^\alpha \prb{X \ge k} = 1 \ee with index $\alpha > 0$, and
$C > 0$ is a normalising constant.   One can interpret  Eq.\ \eqref{jX} as  stating the form of the  probability distribution  for number of juveniles at least on the order of the population size.     If the random  total number of juveniles $(S_N)$ produced in this way is at least the population size $(N)$, then one samples  $N$ juveniles uniformly at random without replacement to form the next generation of  reproducing individuals; if  $S_N < N$ one simply  carries on with  the same  set of individuals.   However,   assuming  $\EE{X} > 1$, one can show that $\prb{S_N < N}$ decays exponentially fast in $N$ \citep{schweinsberg03}.         If $\alpha \ge 2$ the ancestral
process converges to the Kingman-coalescent; if $1 \le \alpha < 2$ the
ancestral process converges to a specific case of a
$\Lambda$-coalescent, where the $\Lambda$ measure in Eq.\
\eqref{lambdabk} is associated with the Beta$(2-\alpha, \alpha)$
probability distribution, i.e.\
\be\label{Fbeta}
    \Lambda(dx) = \one{0< x \le 1} \frac{1}{B(2-\alpha,\alpha)} x^{1 - \alpha}(1-x)^{\alpha - 1}  dx,
\ee
where $B(2-\alpha,\alpha)$ is the beta function $B(a,b) = \Gamma(a)\Gamma(b)/\Gamma(a+b)$, $a,b > 0$ \citep{schweinsberg03}.    This model has been adapted to diploid populations  by                      \cite{BLS15}, where  the resulting coalescent process  is a  four-fold $\Xi$-coalescent on  $(x/4, x/4, x/4, x/4, 0, 0, \ldots)$, where $x$ is a random variate with  the Beta$(2-\alpha,\alpha)$ distribution.  The merger rate (see Eq.\ \eqref{xi}) is then ($1 \le r \le 4$)
\be\label{xibeta}
   \lambda_{b;k_1, \ldots, k_r} = \sum_{\ell = 0}^{ (b - k)\wedge (4-r) } \binom{b-k}{\ell} \frac{ (4)_{r+\ell} }{4^{k+\ell}} \frac{B(k+\ell - \alpha, b-k-\ell + \alpha ) }{B(2-\alpha,\alpha)}
\ee
\citep{Blath2016,BLS15}. The interpretation of Eq.\ \eqref{xibeta} is  that the random   number of diploid  juveniles each  diploid pair of parents  produces  is  governed by  the law in Eq.\ \eqref{jX},   and   each diploid juvenile  is assigned  one chromosome  from each parent (selfing is excluded).   



The model in Eq.\ \eqref{jX} assumes that individuals can produce
arbitrarily large numbers of juveniles. Considering diploid juveniles,
this assumption is probably rather strong, since diploid juveniles are
at least fertilised eggs, and so it is reasonable to suppose that the
number of juveniles surviving to the  particular life stage we are modeling  cannot be
arbitrarily large.  With this in mind, we also consider an adaptation
of the Schweinsberg model, where the random number of juveniles $(X)$
produced by a   given parent pair  is distributed according to
\be\label{jtr}
  \prb{X=k} =   \one{1 \le k \le \phi(N)} \frac{\phi(N+1)^\alpha }{ \phi(N+1)^\alpha - 1 }  \left( \frac{1}{k^\alpha} - \frac{1}{(k+1)^\alpha}  \right) ,
\ee
where $\phi(N)$ is a deterministic strict upper bound on the number of juveniles produced by  any given parent pair (see also \citep{Eldon2018}).   One can follow the calculations in  \citep{schweinsberg03} or \citep{BLS15}  to show  that  if $1 < \alpha < 2$   then  $(k = k_1 + \cdots + k_r)$ then the merger rate (see Eq.\ \eqref{xibeta}) is
\be
   \lambda_{b;k_1, \ldots, k_r} =  \sum_{\ell = 0}^{ (b - k)\wedge (4-r) } \binom{b-k}{\ell} \frac{ (4)_{r+\ell} }{4^{k+\ell}} \frac{B(M; k+\ell - \alpha, b-k-\ell + \alpha ) }{B(M;2-\alpha,\alpha)}
\ee
with $B(z;a,b) := \int_0^z t^{a-1}(1-t)^{b-1}dt$ for  $a,b>0$ and $0< z\le 1$, and
\be
M :=  \frac{K}{K+m} \one{\phi(N) = KN} + \one{\phi(N)/N \to \infty }
\ee
where $K > 0$ is a constant and  $m := \lim_{N\to \infty} \EE{X} = 1 + 2^{1-\alpha}/(\alpha - 1)$ \citep{CDEE2020,AEKKZ2020}.    In other words,  the measure $(\Lambda)$ driving the multiple mergers is of the same form as in Eq.\ \eqref{Fbeta}  with $0 < x \le M$ in the case $1 < \alpha < 2$ and $\phi(N) \ge KN$.   If $\alpha \ge 2$ or $\phi(N)/N \to 0$ then  the  ancestral process converges (in the sense of convergence of  finite-dimensional distributions) to  the Kingman-coalescent \citep{CDEE2020,AEKKZ2020}.



\cite{Becheler2020}  investigate a general framework for  constructing efficient algorithms for general  coalescent processes.   

%assumption can be
%violated in some situations. Lambda coalescents
%%~\citep{donnelly1999particle,schweinsberg2000coalescents}
%%HybridLambda~\citep{zhu2015hybrid}

\begin{comment}%
A coalescent process is a continuous-time Markov process taking values among the
partitions of $\IN := \{1,2, \ldots \}$, such that the restriction to
any finite $n \in \IN $ takes values among partitions of
$[n] := \{1, 2, \ldots, n\}$.  Write $\one{A} = 1$ if $A$ holds, and zero otherwise.   Let $\cP_n$ denote the set of
partitions of $[n]$.  In the classical Kingman-coalescent, the only
possible transitions are the mergers of pairs of blocks (elements of a
partition $\pi \in \cP_n$), one pair at a time.  The $n$ leaves
(corresponding to the sampled DNA sequences) are arbitrarily labelled
from 1 to $n$, and  the blocks of a partition represent the common ancestors
of the labels of each block.    The initial state is  (usually) taken as 
$\{ \{1\}, \ldots, \{n\}\}$, and the final state, i.e.\ when the most 
recent common ancester is reached, as $\{ [n]\}$. A block in a partition of
$[n]$ represents an ancestor of the leaves in the block, i.e.\ the block
$\{i_1, \ldots, i_k\}$ in a given partition of $[n]$ is an ancestor of the
$k$ leaves $i_1, \ldots, i_k \in [n]$, and the leaves  correspond to
arbitrarily labelled DNA sequences in the sample.  
\end{comment}



Mention~\citep{becheler2020occupancy} somewhere.

\subsection*{Ancestral recombination graphs}
The definition of ARGs is vague and confusing.
\cite{minichiello2006mapping} define it as the data structure
rather than the original stochastic
process~\citep{griffiths1991two,griffiths1997ancestral}.
\cite{rasmussen2014genome} have a slightly different approach.
Sometimes the ARG is useful.
The tree sequence output for \msprime\ can be turned into
an ARG using the \texttt{record\_full\_arg} option, where we store
recombination nodes.

We also provide an interface to compute the likelihood of a given ARG
under the~\cite{kuhner2000maximum} model.

\subsection*{Simultaneous bottlenecks}
Overview of what simulataneous bottlenecks literature. What we
do [Konrad].


\subsection*{Seedbank coalescent}
Seedbank coalescent. [If we implement this fill in please Thibaud]

\subsection*{Simulation interface}
The primary interface for \msprime\ is through a Python application
programming interface (API). This has many advantages. One is
that we can grow the interface over time. Another is that
downstream applications such as \texttt{stdpopsim}~\citep{adrion2019community}
can be developed.

\subsection*{Output}
The output of \msprime\ is a ``succinct tree sequence`` (or tree sequence,
for short) using the \tskit\ library. [Quick overview of the output formats
of other simulators, and the perf advantages of tree sequences.]

\subsection*{Development model}
The community development process for \msprime\ is enabled by following an
open source development practise with a strong emphasis on code quality
and correctness. To ensure a consistent style, we require that all code
follows the PEP8 guidelines. Unit tests for all additions are required, and
are automatically run on each pull request on a variety of continuous
integration testing platforms. For new simulation features, statistical tests
must also be added, comparing the distribution of results to either analytical
results or existing simulators.

\subsection*{Computing coalescence rates and lineage location probabilities}
Many popular approaches in population genetics use the distribution of coalescence rates between 
pairs of lineages in one or more populations to infer effective population sizes over time 
\cite{li2011inference,sheehan2013estimating,schiffels2014inferring}
or split times 
and subsequent migration rates between populations 
\cite{wang2020tracking}.
Given a demographic model, the 
\texttt{msprime} DemographyDebugger is able to compute coalescence rates for two or more lineages drawn 
from one or more populations at specified times in the past, providing a ground truth for coalescent rate-based 
inference methods. To compute coalescence rates for a set of times $Ts=(t_0, t_1, \ldots)$ in the 
past, we write
\begin{lstlisting}[frame=single]
dbg = demography.debug()
rates = dbg.coalescence_rate_trajectory(Ts, num_samples)
\end{lstlisting}
where \texttt{num\_samples} is a list with length equal to the number of populations that
specifies the number of sampled lineages from each population. For example, 
to get pairwise rates within the same 
population, this list would contain zeros for all populations except for the population index of interest, 
which would have a $2$. To get the rate of coalescence for two lineages sampled from different 
populations, we would set \texttt{num\_samples} to have ones in those population indices.

More simply, \texttt{msprime} can also compute the probability of a lineage that was in a given
population at some time to be found in other populations (through continuous or mass migration events). 
For a given sample time $t\geq0$ and any list of times $Ts$ in the past, we write
\begin{lstlisting}[frame=single]
prob = dbg.lineage_probabilities(Ts, sample_time=t)
\end{lstlisting}
This function is also used to determine the time intervals that lineages can be found in each population,
given a list of contemporary and/or ancient samples. For a given sampling configuration, by writing
\begin{lstlisting}[frame=single]
lineage_dict = dbg.possible_lineage_locations(samples=samples)
\end{lstlisting}
we get a dictionary with epoch intervals as keys whose values are a list that indicates True or False for 
whether each population could contain ancestral material to our samples.
If no sampling configuration is provided, we assume we sample lineages from each population at time zero.


%%%%%%%%%%%%%%%%%%%%%%%%%%%%%%%%%%%%%%%%%%%%%%%%%%%%%%%%%%%%%%%%%%%%%%%%%%%%%%%%%%%%%
% Results and Discussion: For software, the authors are expected to present a
% few examples that demonstrate the use of the software and benchmarks. Ideally,
% the examples should be simple. We encourage the use of boxes with scripts
% supported by a narrative. Benchmarks carried under more realistic situations
% than the one presented in the examples are also encouraged.
%%%%%%%%%%%%%%%%%%%%%%%%%%%%%%%%%%%%%%%%%%%%%%%%%%%%%%%%%%%%%%%%%%%%%%%%%%%%%%%%%%%%%

\section*{Results and Discussion}
\subsection*{Performance}
We compare \msprime\ with some existing simulators that have a subset of its
features.


\subsection*{Examples}
We show a few examples of running \msprime.

\subsection*{Discussion}
% Simulation is useful, and it's really easy for a single locus.
% This has lead to a culture of developing lots of simulators
% rather than focusing on one
Simulation has played a key role in the development of coalescent theory
and its use in understanding observed patterns of genetic variation.
The basic model can be simulated for $n$ samples in $O(n)$ time
and is straightforward to implement---for example, Hudson provided a fully
functional and highly efficient simulator as a C source code listing in an
appendix to his seminal review paper~\citep{hudson1990gene}. Because of
this ease, the dominant approach to producing simulators for extensions
to this basic model has been to develop them independently. This is
often seen as a learning opportunity, since those developing extensions
to the coalescent must have an excellent understanding of the basic
model in order to develop extensions to the coalescent.
The coalescent with recombination, however, is \emph{not} straightforward
to implement, and numerous different approaches have been
proposed over the
years~\citep{hudson1983properties,griffiths1997ancestral,wiuf1999recombination,
mcvean2005approximating}. It has only recently become clear that it is
possible to implement the model efficiently, and only through the use
of sophisticated data structures and an intricate and subtle
algorithm~\citep{kelleher2016efficient}.
This efficiency, and the ability to handle recombination in a principled
way is vital with today's vast numbers of whole genome samples.

% Making lots of simulators has given us a fragmented and poort
% quality ecosystem.
The approach of developing ``in-house'' simulators
has lead to a huge proliferation of different programs, with
slightly different feature sets and often incompatible interfaces.
It is a fragmented ecosystem with significant issues with software
quality (see, e.g.,~\cite{yang2014critical});
few simulators are actively maintained after publication.
There seems to be little point
in continuing the huge duplication of effort we have seen.
It is not possible to achieve the performance levels of
\msprime\ without investing substantially in software engineering.
As demonstrated here, \msprime\ is extensible and we would argue
that developing this simulator as an open-source community asset
is of more benefit to all.

There is also the welcome development of libraries such as
Quetzal~\citep{becheler2019quetzal} and libsequence~\citep{thornton2014cpp}
which break this pattern of monolithic simulation software.


Thus, development of \msprime\ is not completed and much remains
to be done. For example, one key aim of the 1.0 series is to
improve the mutation models supported which are limited to
simple infinite-sites currently. The ability to simulate
sophisticated mutation models like Seq-Gen~\citep{rambaut1997seq},
Dawg~\citep{cartwright2005dna} and Pyvolve~\citep{spielman2015pyvolve} at the megasample scale would be
useful in many applications. Another key focus of development
will be [WHAT?]

% Finish up with a paragraph saying that there loads of exciting
% new things that simulation can do. We need MORE sims!!!
The ability to infer trees at scale for the first time
in the presence of
recombination~\citep{harris2019database,kelleher2019inferring,
speidel2019method,tang2019genealogy}
opens the possibility for new applications of coalescent simulations.
Also \stdpopsim\ is cool~\citep{adrion2019community}

\section*{Acknowledgments}
JK is supported by the Robertson Foundation. [Others fill in their ACKs please]

\bibliographystyle{plainnat}
\bibliography{paper}

\end{document}
