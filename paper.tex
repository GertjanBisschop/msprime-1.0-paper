\documentclass{article}
\usepackage[round]{natbib}


% local definitions
\newcommand{\msprime}[0]{\texttt{msprime}}
\newcommand{\tskit}[0]{\texttt{tskit}}
\newcommand{\ms}[0]{\texttt{ms}}
\newcommand{\scrm}[0]{\texttt{scrm}}
\newcommand{\stdpopsim}[0]{\texttt{stdpopsim}}


\begin{document}

\title{msprime 1.0: an efficient and extensible coalescent simulation framework}
\author{Author list to be filled in
}
% \address{

% \section{Contact:} \href{jerome.kelleher@bdi.ox.ac.uk}{jerome.kelleher@bdi.ox.ac.uk}

\maketitle

% Abstract: ~ 150 words.  Concise summary of the results.

% Key points in abstract.
% 1) Coalescent simulation is a vital tool
% 2) Coalescent with recombination is hard
% 3) Large sample sizes are crucial today
% 4) Msprime has lots of features and uses a community development model

\begin{abstract}
Coalescent simulation is a key tool in population genetics and
and has been integral to coalescent theory since its earliest days.
Because of the ease at which the basic model can be simulated,
a large number of different simulators have been developed. However,
the coalescent with recombination is far more challenging to simulate,
and, until recently, it was not possible to simulate efficiently.
The \msprime\ software has revolutionised
coalescent simulation, making it possible to simulate millions
of whole genomes for the first time. We summarise the many features
of \msprime\ 1.0, which is built around a core model of efficiently
implementing recombination via the succinct tree sequence data
structure. We advocate a community oriented open-source development
process as a way to reduce duplication of effort and increase
the quality of simulation software.
\end{abstract}

% Keywords: For software, include the relevant models, algorithms and language;
\textbf{Keywords:} Coalescent simulation, Python

%%%%%%%%%%%%%%%%%%%%%%%%%%%%%%%%%%%%%%%%%%%%%%%%%%%%%%%%%%%%%%%%%%%%%%%%%%%%%%%%%%%%%
% Introduction: Limit of 500 words. The introduction should describe the
% significance of the software or resource presented, discuss novelty, provide an
% overview of the software or resource, and describe how researchers can access it.
%%%%%%%%%%%%%%%%%%%%%%%%%%%%%%%%%%%%%%%%%%%%%%%%%%%%%%%%%%%%%%%%%%%%%%%%%%%%%%%%%%%%%

\section*{Introduction}

The coalescent
process~\citep{kingman1982coalescent,hudson1983testing,tajima1983evolutionary}
models the ancestry of set of sampled genomes. Given some sampled
genomes, the coalescent provides a mathematical description of the
genealogical tree relating these to each other. Because of its
elegance and simplicity, the coalescent is now central to much of
population genetics and
genomics~\citep{hudson1990gene,hein2004gene,wakely2008coalescent}.
Simulation has been a vital
tool in coalescent theory since its earliest days~\citep{hudson1983testing},
and dozens of different simulation tools have been
developed over the decades~\citep{carvajal2008simulation,liu2008survey,
arenas2012simulation,yuan2012overview,hoban2012computer}.
Two developments of recent years, however, have made the majority
of these simulators of little relevance to modern datasets.
% Some citations here? Darwin tree of life maybe?
Firstly, whole genome sequencing technology---and in particular fourth-generation
sequencing technologies---have made chromosome-level assemblies
commonplace. Thus, regarding input data as a series of unlinked loci
(a previously reasonable assumption), is no longer defensible and
recombination must be fully accounted for. While the
single-locus coalescent can be simulated in constant
time~\citep{hudson1990gene}, the coalescent with recombination is
far more challenging to simulate.
% Sentence is too long...
Programs such as Hudson's classical \ms~\citep{hudson2002generating}
can only simulate short segments under the influence of recombination,
and, until recently, we had to resort to
approximate models of recombination~\citep{mcvean2005approximating,staab2014scrm}
if we wished to simulate whole chromosomes.

The second development that has made the majority of
present day simulators inapplicable to modern data sets is the
exponential increase in sample size. Human datasets such
UK Biobank~\citep{bycroft2018genome} and
gnomAD~\citep{karczewski2019variation} now contain hundreds of
thousands of genomes, and there is every reason to believe that
datasets of millions will soon be commonplace~\citep{stephens2015big}.
Classical simulators such as \ms\ and even fast approximate simulators
such as \scrm\ simply cannot cope with this scale. Even if we
spend months of CPU time on running simulations with millions of samples,
the output would require terabytes of storage space to store and
days of CPU time to parse~\citep{kelleher2016efficient}.

The \msprime\ simulator~\citep{kelleher2016efficient,kelleher2020coalescent}
has profoundly changed
what is possible. Through the use of efficient data structures, it is
now possible to simulate millions of whole chromosomes, without resorting
to approximations. The ``succinct tree
sequence'' data
structure~\citep{kelleher2016efficient,kelleher2018efficient,kelleher2019inferring},
introduced as part of \msprime\, makes it possible to store simulations
of millions of whole chromosomes in a few gigabytes, several orders
of magnitude smaller than the standard
Newick~\citep{felsenstein1989phylip} formats. This data structure has
also lead to major advances in forwards-time
simulation~\citep{kelleher2018efficient,haller2018tree},
ancestry inference~\citep{kelleher2019inferring}
and calculation of population genetic statistics~\citep{ralph2019efficiently}.
Through a rigorous open-source community development process,
\msprime\ has gained a large number of features since its introduction,
making a highly efficient and flexible platform for population
genetic simulation.
% This is a bit flat --- anything else to say at the end here?
This paper marks the release of \msprime\ 1.0.

% JK: Earlier notes on this para
% - Msprime 1.0 combines the features of many different simulators and is
%   far more efficient than all of them. The combination of a Python API,
%   an extremely efficient storage format, and a well-engineered and
%   extensible codebase changes the game. Stdpopsim builds on this base
%   to provide easy access to standard models of population genetics.
%


%%%%%%%%%%%%%%%%%%%%%%%%%%%%%%%%%%%%%%%%%%%%%%%%%%%%%%%%%%%%%%%%%%%%%%%%%%%%%%%%%%%%%
% Methods: For software, the authors should describe key elements of the algorithms
% implemented, and are expected to provide both the source code as well as
% examples illustrating the use of the software.
%%%%%%%%%%%%%%%%%%%%%%%%%%%%%%%%%%%%%%%%%%%%%%%%%%%%%%%%%%%%%%%%%%%%%%%%%%%%%%%%%%%%%

\section*{Methods}

In this section we describe the main features of \msprime\ 1.0.

\subsection*{Recombination and gene conversion}
Recombination is implemented in \msprime\ via
Hudson's algorithm, which
works backwards in time tracking the
through the effects of common ancestor and recombination events
~\citep{hudson1983properties,hudson1990gene,kelleher2016efficient}. Common
ancestor events combine the ancestral material of two lineages, and may
 result in coalescences in the marginal trees. Recombination events
split the ancestral material for some lineage, creating two independent
lineages. Using the appropriate data structures~\citep{kelleher2016efficient},
this process is much more efficient to simulate than either
methods based on the Ancestral Recombination Graph
formulation~\citep{griffiths1991two,griffiths1997ancestral}
or the left-to-right approach~\citep{wiuf1999recombination,wiuf1999ancestry}.


We also support variable recombination rates along a chromosome.
This allows us to model both recombination hotspots like
\texttt{msHOT}~\citep{hellenthal2007mshot} and also empirical
recombination maps like those supported by X Y Z.
(What about ~\cite{wang2014new}?)
We have also implemented gene
conversion following the model defined by~\cite{wiuf2000coalescent}
and used in \ms.
% % Uncomment this sentence, when the corresponding issue in the msprime github repo is resolved.
% Furthermore, msprime is now able to simulate gene conversion events that are
% initiated proportional to the local recombination rate specified by the recombination map.

Gene conversion is particularly useful to model homologous recombination in bacterial evolution.
We compare the performance of \msprime\ with gene conversion to \ms\ and
the whole bacterial genome simulator SimBac~\citep{brown2016simbac}.
Further features are planned to model bacterial evolution. In particular, we
aim to include bacterial gene transfer and utilize the succinct tree sequence structure
to represent the resulting ancestral gene transfer graph~\citep{baumdicker2014AGTG}.

\subsection*{SMC approximations}

The SMC~\citep{mcvean2005approximating} and SMC'~\citep{marjoram2006fast}
models are supported.
The performance of msprime has made the SMC
approximation largely unnecessary, at least when simulating human-like
recombination rates. The SMC is an important analytical approximation
forming the basis for several important inference methods
methods~\citep{
li2011inference,
harris2013inferring,
sheehan2013estimating,
schiffels2014inferring,
carmi2014renewal,
rasmussen2014genome,
zheng2014bayesian,
terhorst2017robust},
so simulations are still useful to examine  how good the approximation
is.

The SMC has been shown to be a good approximation to the
coalescent with recombination for samples of less than
four~\citep{hobolth2014markovian,wilton2015smc}
and in simulations, produces patterns of LD indistinguishable from the
full coalescent~\citep{mcvean2005approximating,marjoram2006fast}. However,
LD is a crude approximation of the correlation structure of trees
and it is not clear how good the SMC is for larger sample sizes.


\subsection*{Population structure}
We support all the stuff that ms does.
Demographic events where we change
the properties of the population.

We provide a debugger interface which allows the user to see the
various population sizes and growth rates over time. This includes
a way to compute coalescence rate trajectory, which provides a useful
ground truth when comparing with methods that estimate population
size over time~\citep{adrion2019community}.

The interface
is very flexible and general. We also reading in population tree
descriptions as produced by programs such as
StarBeast~\citep{heled2009bayesian}.

We support tracking lineages through time by associating a population
with each node, and having migration records. We also have
census events, which will record the location of every lineage
at a particular time.

\subsection*{Selection}
Selection was added to the coalescent ... Lots of simulators exist:
SelSim~\citep{spencer2004selsim}
mbs~\citep{teshima2009mbs},
msms~\citep{ewing2010msms},
% What about Cosi, did it simulate selection?
Cosi2~\citep{shlyakhter2014cosi2},
and discoal~\citep{kern2016discoal}.

We have implemented the basic infrastructure exists for the structured
coalescent, which makes adding support for, e.g.,
inversions straightforward~\citep{peischl2013sequential}


\subsection*{Discrete time Wright-Fisher}
The coalescent is a good approximation for ancient history, but
is bad for the recent past and in particular when we have large
sample
sizes~\citep{wakeley2012gene,bhaskar2014distortion,nelson2020accounting}.
Describe basic
approach [DOM, can you fill in please].
See ~\cite{nelson2020accounting} for more details.
We compare the performance of this simulation in Fig X
with ARGON~\citep{palamara2016argon}, a specialised DTWF simulator.

We support changing the simulation model, so that we can easily
run hybrid simulations like described by~\cite{bhaskar2014distortion}
This ia a powerful approach because it lets us combine any
number of different simulation models.

\subsection*{Integration with forwards-time simulations}
Forward simulations can also use tree
sequences~\citep{kelleher2018efficient,haller2018tree}. In particular,
SLiM 3.0~\citep{haller2019slim} and fwdpp~\citep{thornton2014cpp} can now output tree sequences.
Msprime can take tree sequences simulated by these forward simulators
as input and complete the simulation (``recapitation'').

\subsection*{Multiple merger coalescents}
Kingman's coalescent assumes that all mergers are binary. This
assumption can be violated in some situations. Lambda coalescents
~\citep{donnelly1999particle,schweinsberg2000coalescents}
HybridLambda~\citep{zhu2015hybrid}
We support X Y Z.

Mention~\citep{becheler2020occupancy} somewhere.

\subsection*{Ancestral recombination graphs}
The definition of ARGs is vague and confusing.
\cite{minichiello2006mapping} define it as the data structure
rather than the original stochastic
process~\citep{griffiths1991two,griffiths1997ancestral}.
\cite{rasmussen2014genome} have a slightly different approach.
Sometimes the ARG is useful.
The tree sequence output for \msprime\ can be turned into
an ARG using the \texttt{record\_full\_arg} option, where we store
recombination nodes.

We also provide an interface to compute the likelihood of a given ARG
under the~\cite{kuhner2000maximum} model.

\subsection*{Simultaneous bottlenecks}
Overview of what simulataneous bottlenecks literature. What we
do [Konrad].


\subsection*{Seedbank coalescent}
Seedbank coalescent. [If we implement this fill in please Thibaud]

\subsection*{Simulation interface}
The primary interface for \msprime\ is through a Python application
programming interface (API). This has many advantages. One is
that we can grow the interface over time. Another is that
downstream applications such as \texttt{stdpopsim}~\citep{adrion2019community}
can be developed.

\subsection*{Output}
The output of \msprime\ is a ``succinct tree sequence`` (or tree sequence,
for short) using the \tskit\ library. [Quick overview of the output formats
of other simulators, and the perf advantages of tree sequences.]

\subsection*{Development model}
The community development process for \msprime\ is enabled by following an
open source development practise with a strong emphasis on code quality
and correctness. To ensure a consistent style, we require that all code
follows the PEP8 guidelines. Unit tests for all additions are required, and
are automatically run on each pull request on a variety of continuous
integration testing platforms. For new simulation features, statistical tests
must also be added, comparing the distribution of results to either analytical
results or existing simulators.


%%%%%%%%%%%%%%%%%%%%%%%%%%%%%%%%%%%%%%%%%%%%%%%%%%%%%%%%%%%%%%%%%%%%%%%%%%%%%%%%%%%%%
% Results and Discussion: For software, the authors are expected to present a
% few examples that demonstrate the use of the software and benchmarks. Ideally,
% the examples should be simple. We encourage the use of boxes with scripts
% supported by a narrative. Benchmarks carried under more realistic situations
% than the one presented in the examples are also encouraged.
%%%%%%%%%%%%%%%%%%%%%%%%%%%%%%%%%%%%%%%%%%%%%%%%%%%%%%%%%%%%%%%%%%%%%%%%%%%%%%%%%%%%%

\section*{Results and Discussion}
\subsection*{Performance}
We compare \msprime\ with some existing simulators that have a subset of its
features.


\subsection*{Examples}
We show a few examples of running \msprime.

\subsection*{Discussion}
% Simulation is useful, and it's really easy for a single locus.
% This has lead to a culture of developing lots of simulators
% rather than focusing on one
Simulation has played a key role in the development of coalescent theory
and its use in understanding observed patterns of genetic variation.
The basic model can be simulated for $n$ samples in $O(n)$ time
and is straightforward to implement---for example, Hudson provided a fully
functional and highly efficient simulator as a C source code listing in an
appendix to his seminal review paper~\citep{hudson1990gene}. Because of
this ease, the dominant approach to producing simulators for extensions
to this basic model has been to develop them independently. This is
often seen as a learning opportunity, since those developing extensions
to the coalescent must have an excellent understanding of the basic
model in order to develop extensions to the coalescent.
The coalescent with recombination, however, is \emph{not} straightforward
to implement, and numerous different approaches have been
proposed over the
years~\citep{hudson1983properties,griffiths1997ancestral,wiuf1999recombination,
mcvean2005approximating}. It has only recently become clear that it is
possible to implement the model efficiently, and only through the use
of sophisticated data structures and an intricate and subtle
algorithm~\citep{kelleher2016efficient}.
This efficiency, and the ability to handle recombination in a principled
way is vital with today's vast numbers of whole genome samples.

% Making lots of simulators has given us a fragmented and poort
% quality ecosystem.
The approach of developing ``in-house'' simulators
has lead to a huge proliferation of different programs, with
slightly different feature sets and often incompatible interfaces.
It is a fragmented ecosystem with significant issues with software
quality (see, e.g.,~\cite{yang2014critical});
few simulators are actively maintained after publication.
There seems to be little point
in continuing the huge duplication of effort we have seen.
It is not possible to achieve the performance levels of
\msprime\ without investing substantially in software engineering.
As demonstrated here, \msprime\ is extensible and we would argue
that developing this simulator as an open-source community asset
is of more benefit to all.

There is also the welcome development of libraries such as
Quetzal~\citep{becheler2019quetzal} and libsequence~\citep{thornton2014cpp}
which break this pattern of monolithic simulation software.


Thus, development of \msprime\ is not completed and much remains
to be done. For example, one key aim of the 1.0 series is to
improve the mutation models supported which are limited to
simple infinite-sites currently. The ability to simulate
sophisticated mutation models like Seq-Gen~\citep{rambaut1997seq},
Dawg~\citep{cartwright2005dna} and Pyvolve~\citep{spielman2015pyvolve} at the megasample scale would be
useful in many applications. Another key focus of development
will be [WHAT?]

% Finish up with a paragraph saying that there loads of exciting
% new things that simulation can do. We need MORE sims!!!
The ability to infer trees at scale for the first time
in the presence of
recombination~\citep{harris2019database,kelleher2019inferring,
speidel2019method,tang2019genealogy}
opens the possibility for new applications of coalescent simulations.
Also \stdpopsim\ is cool~\citep{adrion2019community}

\section*{Acknowledgments}
JK is supported by the Robertson Foundation. [Others fill in their ACKs please]

\bibliographystyle{plainnat}
\bibliography{paper}

\end{document}
